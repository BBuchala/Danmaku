% !TeX spellcheck = pl_PL
\documentclass[a4paper,twoside]{article}
\usepackage{polski}
\usepackage[utf8]{inputenc}
\usepackage{graphicx}
\usepackage{amsmath}

\usepackage[unicode, bookmarks=true]{hyperref} %do zakładek
\usepackage{tabto} % do tabulacji
\NumTabs{6} % globalne ustawienie wielkosci tabulacji
\usepackage{array}
\usepackage{multirow}
\usepackage{array}
\usepackage{dcolumn}
\usepackage{bigstrut}
\usepackage{color}
\usepackage[usenames,dvipsnames]{xcolor}
\usepackage{svg}
\usepackage{xfrac}
\usepackage{floatrow}


\usepackage{multirow,tabularx}
\newcolumntype{Y}{>{\centering\arraybackslash}X}
\renewcommand{\arraystretch}{2}

% === Reset inkrementacji sekcji przy nowym parcie === %
\usepackage{titlesec}

\makeatletter
\@addtoreset{section}{part}
\makeatother
\titleformat{\part}[display]
{\normalfont\LARGE\bfseries\centering}{}{0pt}{}


\setlength{\textheight}{24cm}
\setlength{\textwidth}{15.92cm}
\setlength{\footskip}{10mm}
\setlength{\oddsidemargin}{0mm}
\setlength{\evensidemargin}{0mm}
\setlength{\topmargin}{0mm}
\setlength{\headsep}{5mm}

\setlength{\textfloatsep}{10pt plus 1.0pt minus 2.0pt}


\begin{document}
\bibliographystyle{plain}

% ************************************************************
% --- Strona tytułowa
% ************************************************************
\begin{titlepage}
	\begin{table}[htbp]
		\centering
		\begin{tabular}{|c|c|c|c|c|c|c|}
			\hline
			\multicolumn{7}{|c|}{\textbf{{\LARGE Projekt programistyczny}}} \bigstrut\\[4pt]
			\hline
			Rok akademicki & Termin & Rodzaj studiów & Kierunek & Prowadzący & Grupa & Sekcja \bigstrut\\
			\hline
			\multicolumn{1}{|c|}{\multirow{2}[4]{*}{{\large 2014/2015}}} & \multicolumn{1}{c|}{{\large Wtorek}} & \multicolumn{1}{c|}{\multirow{2}[4]{*}{{\large SSI}}} & \multicolumn{1}{c|}{\multirow{2}[4]{*}{{\large INF}}} & \multicolumn{1}{c|}{\multirow{2}[4]{*}{\begin{tabular}{@{}c@{}}{\large dr inż.} \\[-9pt] {\large Arkadiusz}\\[-9pt] {\large Biernacki}\end{tabular}}} & \multicolumn{1}{c|}{\multirow{2}[4]{*}{{\large GKiO3}}} & \multicolumn{1}{c|}{\multirow{2}[4]{*}{{\large 1}}} \bigstrut\\
			\cline{2-2}    \multicolumn{1}{|c|}{} & \multicolumn{1}{c|}{{\large 12:45 - 15:00}} & \multicolumn{1}{c|}{} & \multicolumn{1}{c|}{} & \multicolumn{1}{c|}{} & \multicolumn{1}{c|}{} & \multicolumn{1}{c|}{} \bigstrut\\
			\hline
		\end{tabular}%
	\end{table}%
	
	\centering
	\includegraphics[width=0.6\textwidth]{./images/logo.png}
	\\\vspace{10mm}
	\textbf{{\huge Karta projektu}}\\\vspace{5mm}
	\textbf{{\Huge Danmaku Shooter}}
	\\
	\vfill
	\begin{flushright}
			{\Large \textbf{Skład sekcji}:}\\
		\begin{tabular}{rr}
			{\Large Buchała} & {\Large Bartłomiej}\\[-3pt]
			{\Large Forczmański} & {\Large Mateusz}\\[-3pt]
			{\Large Motyka} & {\Large Marek}\\[-3pt]
			{\Large Wudecki} & {\Large Wojciech}
		\end{tabular}
	\end{flushright}
	
\end{titlepage}



% ************************************************************
% --- Strona z założeniami
% ************************************************************
\part{\huge \textbf{Krótki opis aplikacji}}
Shoot' em up (w skrócie zwany shmup) jest gatunkiem gier akcji wywodzącym się w prostej linii od gier typu \textit{Space Invaders} lub \textit{River Raid}. Kontrolowana przez gracza postać (np. statek) w pojedynkę stawia czoło przeciwnikom, niszcząc ich za pomocą wystrzeliwanych pocisków, jednocześnie unikając ich ataków. Podgatunek shmupów, zwany danmaku (z jap. \textit{ściana pocisków} lub \textit{piekło pocisków}) kładzie większy nacisk na omijanie wrogich ataków, niż na ofensywie. Przykładowymi danmaku są np. \textit{Ikaruga} czy większość gier z uniwersum \textit{Touhou}. \\\\
Obiekt gracza, wrogowie oraz pociski będą widoczne z góry w prostokątnym oknie gry, po którym będzie mógł poruszać się gracz.\\\\
Gra powstawać będzie jako projekt łączony z przedmiotów Projekt Programistyczny i Grafika Komputerowa.

% TO DO: Insert 9000 godzin w Paincie. % 

% Piekło pocisków brzmi jak Mikołów. Jesteś sobie taki biedny ty, a obok Ciebie Raku i Korda i Ci co chwilę pociskają ;_; %

% Zostawiam póki co takie wykastrowane, dopóki nie ustalimy więcej szczegółów. W ostateczności można tak zostawić, bo to ma być króciutki zarys. %

\newpage

\part{\huge \textbf{Założenia projektowe}}

\section{Narzędzie do modelowania UML}

{\Large Enterprise Architect} \\
Jest to narzędzie, który poznaliśmy semestr wcześniej na projekcie z przedmiotu Inżynieria Oprogramowania. Ponieważ ww. przedmiot kładł bardzo duży nacisk na modelowanie diagramów, większość z nas jest z nim dobrze zaznajomiona. Przejrzystość tego programu może być niezwykle przydatna podczas tworzenia np. diagramu przypadków użycia. Równolegle wykorzystujemy EA do stworzenia diagramów UML dla projektu z przedmiotu Bazy Danych II.

\section{System kontroli wersji}

{\Large GitHub} \\
Jest to bardzo powszechnie stosowane narzędzie wśród deweloperów pracujących w grupach. Jego zaletami jest m.in. klient aplikacji na system operacyjny Windows, łatwy do przyswojenia i intuicyjny interfejs i łatwość w zarządzaniu starszymi wersjami. Podobnie jak w przypadku Enterprise Architecta, większość z nas miała już wcześniej styczność z GitHubem i korzysta z niego od pewnego czasu.

\section{Narzędzi pracy grupowej}
% Takie cuś jest na platformie do opisania. Chyba chodzi o komunikatory. GG, Facebook, Skype? Wuda jako środek transportu do kumpla się liczy? :P %
\section{Środowisko programistyczne}
{\Large Microsoft Visual Studio 2012}
\section{Język programowania}
{\Large C++ / C\#}
\section{Biblioteka graficzna}
{\Large DirectX / OpenGL}
\section{Platforma programistyczna}
{\Large .NET Framework}








\end{document}