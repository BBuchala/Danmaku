% !TeX spellcheck = pl_PL
% -- W niniejszym rozdziale można umieścić komentarze na temat pracy nad programem i inne
% -- spostrzeżenia. Można tu ustosunkować się do założeń umieszczonych w analizie zadania i stwierdzić,
% -- czy oczekiwania były trafne (np. że przewidywany algorytm okazał się odpowiedni do takiego
% -- charakteru/ takiej ilości danych lub że przyjęte rozwiązanie zbyt ograniczyło możliwości programu i
% -- że poszczególne moduły należałoby rozwijać). Rozdział ten nie wpływa na merytoryczną ocenę (o ile
% -- całe, zdefiniowane w porozumieniu z prowadzącym zadanie zostało zrealizowane).
\newpage
\part{\huge \textbf{Wnioski}}
	W czasie pracy nad programem spotkaliśmy się z całą gamą różnorakich problemów. Większość z nich wynikała z naszej niewiedzy na tematów szczegółów grafiki komputerowej, i wielokrotnie musieliśmy zaznajomić się z technikami programowania lub matematycznymi podstawami, żeby wykonać jakieś zadanie. Jednakże właśnie z tego powodu nasz program na swoje niedoskonałości, ponieważ założenia okazały się nieoptymalne.
	\begin{itemize}
		\item Jednym z nich była realizacja krzywej Beziera. Wszystkie algorytmy dotyczące samej krzywej były efektywne i poprawne, jednak struktura naszego programu i nasze algorytmy poruszania się sprawiły, że zastosowanie tego rodzaju krzywej było niezbyt optymalne.
		\item Zaczynając obcowanie z DirectXem zauważyliśmy, że rysowanie sprajtów wymaga podania pozycji, gdzie mają zostać pokazane. Ta pozycja jest jednocześnie lewym górnym rogiem sprajta. Idąc tą logiką zdecydowaliśmy, że pozycja obiektu, przechowywana w klasie GameObject, też będzie wskazywała na ów lewy, górny róg. W czasie późniejszej pracy okazało się, że znacznie częściej potrzebujemy punktu środkowego obiektu niż jego rogu, więc po czasie zdecydowaliśmy się to zmienić (zwłaszcza po zauważeniu, że profesjonalne silniki również tak robią). Wymagało to od nas dużego nakładu pracy, której nie powinno być.
		\item W czasie pracy pojawił się problem zarządzania zasobami, a dokładniej sprajtami. W naszych pierwszych krokach każdy obiekt gry posiadał osobny dla siebie sprajt. Nawet jeżeli wszystkie pociski we wzorze posiadały ten sam sprajt, każdy miał osobno przydzieloną pamięć. Powodowało to znaczne opóźnienia zarówno we wczytywaniu jak i podczas samej gry. Aby to rozwiązać utworzyliśmy typ klas \textit{Resource}, które mapują i przechowują wskaźniki do utworzonych sprajtów. Każdy obiekt otrzymuje potrzebny wskaźnik, za którego pośrednictwem przekazuje informację, w którym miejscu sprajt należy narysować. Dzięki temu każdy sprajt tworzony jest tylko raz.
		\item Wszystkie podstawowe zagadnienia, jakie zrealizowaliśmy w ramach projektu (tory pocisków, ruch fizyczny, przekształcenia) udało się w pełni wykonać i wszystkie nasze implementacje działają płynnie w trakcie działania gry.
		\item Bardziej zaawansowane zagadnienia, jak wykrywanie kolizji i tworzenie krzywych Beziera okazały się bardziej złożone niż nam się zdawało i wymagały od nas więcej planowania i myślenia nad wydajnością. Algorytm wykrywania kolizji musiał zostać kilka raz zmieniony, jednak jego ostateczna forma działa poprawnie. Żeby przyspieszyć jeszcze bardziej wykonywania testów, należałoby zmienić i bardziej poprawić strukturę naszego programu.
		\item Wykorzystywane przez nas narzędzia świetnie się sprawdziły w trakcie tworzenia gry.
	\end{itemize}
		