% !TeX spellcheck = pl_PL
\newpage
\part{\huge \textbf{Analiza zadania}}
% --- Analiza zadania z dokumentu projektowego, poszerzoną o jasno wyszczególnione elementy, które podczas pracy zostały dodane/usunięte/uległy zmianie (z uzasadnieniem)
	\section{Sprawdzanie kolizji - okrąg czy elipsa?}
		\indent \indent W trakcie naszej pracy nad programem początkowo wykorzystywaliśmy okrąg do sprawdzania kolizji. Wysokość i szerokość wszystkich sprajtów była w przybliżeniu równa, więc punkty środkowe okręgu i sprajta były równe, a promieniem był krótszy z boków. Jednak po czasie pojawiły się kształty (np. bomba), dla których obsługa kolizji poprzez okrąg wyglądałaby nieatrakcyjnie. Wówczas postanowiliśmy wykorzystać elipsę. Okrąg jest jedynie specjalnym przypadkiem elipsy o równych półosiach. Gdy obiekt miał mieć hitbox w kształcie elipsy, dłuższy bok definiował długość jednej półosi, a krótszy drugiej.\\\\
		\indent Rozwiązanie to okazało się bardzo mało efektywne - do wykrywania kolizji potrzebny były dodatkowy parametr, kąt pomiędzy obiektami. Dla jednego testu należało obliczyć ten kąt, następnie obliczyć długość promienia elipsy dla tego kąta u obu obiektów i dopiero wtedy była znane zajście kolizji. Dla hitboxów o równych osiach powodowało to mnóstwo niepotrzebnych obliczeń. Nawet pominięcie ich w przypadku równych półosi wymagało obliczania i przekazania kąta, który dla każdej elipsy był niezbędny.\\\\
		W takiej sytuacji mieliśmy dwie możliwości:
		\begin{enumerate}
			\item \textit{Zmiana elipsy wieloma okręgami} - eliminuje wykorzystanie elipsy, jednak dla każdego obiektu jest potrzebne zdefiniowanie ile okręgów potrzebuje, w którym miejscu i o jakiej wielkości, tak, by dopasować je do kształtu sprajta. Projektowanie ciała okręgów byłoby niezwykle niewygodne bez wsparcia designerskiego, którego nasza gra nie miała. 
			\item \textit{Osobne klasy dla okręgu i elipsy} - okręgu potrzebowałyby jak najmniejszej liczby danych i obliczeń, a elipsa pozostałaby obsługiwana. Wadą jest naruszenie zasad polimorfizmu - przy każdym teście okręgi nie potrzebują kąta między obiektami, więc przekazywanie ich jest zbędne i prowadzi do niepotrzebnych operacji.
		\end{enumerate}
	\section{Sprawdzanie kolizji - zastrzyk zależności}
		\indent \indent Rozwiązaniem problemu opisanego w powyższym rozdziale okazał się wzorzec projektowy, zastrzyk zależności (\emph{dependency injection}). Realizujemy go w ten sposób, że przy każdym teście kolizji, do jednego hitboxa wstrzykujemy drugi - pierwszy zna swoje metody działania, rzutuje drugi na odpowiedni typ i sprawdza kolizję zwracając true lub false.\\\\
		\indent Rozwiązanie jest skuteczne, ponieważ jeżeli w teście kolizji znajduje się elipsa, to pobiera one dane jakiej jej potrzeba. Jeżeli nie, test wykonywany jest najmniejszym możliwym kosztem. Sprawdzaniem, z którym typem hitboxa mamy do czynienia zajmuje się mechanizm RTTI.