% !TeX spellcheck = pl_PL
% --- Specyfikacja wewnętrzna jest dokumentacją techniczną powstałego oprogramowania.
% --- Umieścić link do automatycznie wygenerowanej dokumentacji na podstawie komentarzy.
% --- Można również opisać ciekawsze/ważniejsze elementy kodu źródłowego (funkcje, obiekty, algorytmy)
\newpage
\part{\huge \textbf{Specyfikacja wewnętrzna}}
	\section{Dokumentacja techniczna}
		Można ją znaleźć tutaj: 
		
	\section{Wybrane funkcje}
		\subsection{Przekształcenia afiniczne}
			\begin{lstlisting}[language=C++]
				void Hitbox::Translate( D3DXVECTOR2 const & translate )
				{
					*_position += translate;
				};
			\end{lstlisting}
			\begin{lstlisting}[language=C++]
				void GameObject::Scale( float const scale )
				{
					this->scale *= scale;
					this->hitbox->Scale( scale );
				};
			\end{lstlisting}
			\begin{lstlisting}[language=C++]
				void Trajectory::GetRotation( D3DXVECTOR2 & pos, float const theta )
				{
					D3DXMATRIX mat; 
					mat._11 = cos(theta);	mat._12 = sin(theta);	mat._13 = 0.0f;		mat._14 = 0.0f;
					mat._21 = -sin(theta);	mat._22 = cos(theta);	mat._23 = 0.0f;		mat._24 = 0.0f;
					mat._31 = 0.0f;			mat._32 = 0.0f;			mat._33 = 1.0f;		mat._34 = 0.0f;
					mat._41 = 0.0f;			mat._42 = 0.0f;			mat._43 = 0.0f;		mat._44 = 1.0f;
					D3DXVec2TransformCoord(&pos, &pos, &mat);
				};
			\end{lstlisting}
	
		\subsection{Sprawdzenie kolizji gracza z pociskami}
			\begin{lstlisting}[language=C++]
				for (PatternQue::const_iterator p_it = _savedPatterns.begin(); p_it != _savedPatterns.end(); ++p_it)
				{
					std::deque<EnemyBullet*> * ep = (*p_it)->GetBullets();
					std::deque<EnemyBullet*>::const_iterator eb_it = ep->begin();
					{
						while (eb_it != ep->end())
						{
							if ((*eb_it)->GetHitbox()->TestCollision(player->GetHitbox()))
							{
								eb_it = ep->erase(eb_it);	// usunięcie pocisku z kolejki
								this->MakePlayerLoseLife();
								return;
							}
							if (eb_it != ep->end())
								eb_it++;
						}
					}
				}
			\end{lstlisting}
			\begin{lstlisting}[language=C++]
				bool HitboxCircle::TestCollision(Hitbox * collider, USHORT grazeDistance)
				{
					HitboxCircle * hCircle = dynamic_cast<HitboxCircle*>(collider);
					if (hCircle != NULL)
					{
						float distance = Vector::Length(*_position, collider->GetPosition());
						if (distance <= _radius + hCircle->GetRadius() + grazeDistance)
						{
							return true;
						}
						return false;
					}
					HitboxElipse * hElipse = dynamic_cast<HitboxElipse*>(collider);
					if (hElipse != NULL)
					{
						float distance = Vector::Length(hElipse->GetPosition(), *_position);
						float angle = Vector::Angle( *_position, hElipse->GetPosition());
						float radialLength = hElipse->GetRadialLength(angle);
						float maxDistance = _radius + radialLength + grazeDistance;
						if (distance <= maxDistance)
						{
							return true;
						}
						return false;
					}
					return false;
				};
			\end{lstlisting}
		\subsection{Obliczanie położenia na krzywej Beziera}
			\begin{lstlisting}[language=C++]
				D3DXVECTOR2 TrajectoryBezier::GetPosition( float const dis )
				{
					// znormalizowanie wartości
					float distance;
					if (_loopFlag)
						distance = fmod(dis > totalLength || dis < 0.0f ? fmod(dis + totalLength, totalLength) : dis, totalLength);
					else
						distance = dis;
					float tt = (float) distance / this->totalLength;
					// zapisanie obecnej tablicy punktów
					PointVector pointSave = PointVector(this->point);
					// Obliczenie obecnego punktu
					for (size_t k = 1; k < pointSave.size(); ++k)
					{
						for (size_t i = 0; i < pointSave.size() - k; ++i)
						{
							for	(int j = 0; j < 2; ++j)
							{
							pointSave[i][j] = (1 - tt) * pointSave[i][j] + tt * pointSave[i + 1][j];
							}
						}
					}
					return pointSave[0];
				};
			\end{lstlisting}