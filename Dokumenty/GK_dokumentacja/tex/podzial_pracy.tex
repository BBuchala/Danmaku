% !TeX spellcheck = pl_PL
\newcolumntype{Y}{>{\centering\arraybackslash}X}
\renewcommand{\arraystretch}{1.5}
\newpage
\part{\huge \textbf{Podział pracy}}
	\begin{center}
		\begin{tabular}{|p{0.45\textwidth}|p{0.45\textwidth}|}
			\hline {\large \textbf{1.~Buchała Bartłomiej}}
			&  {\large \textbf{2.~Forczmański Mateusz}} \\
			\hline
			\vspace{-5mm}
			\begin{itemize}
				\item Interfejs graficzny i wyświetlanie danych
				\item Ruch gracza po planszy
				\item Strzelanie pociskami przez gracza
				\item Implementacja wzorców pocisków gracza
				\item Wystrzelenie bomby
				\item Narysowanie sprajtów pocisków gracza\newline i bomby
				\item Umożliwienie pociskom ruchu po wybranych torach
				\item Usuwanie pocisków z pamięci gdy znajdą się poza planszą
				\item Sprawne przechodzenie pomiędzy planszami
				\item Zapisywanie uzyskanego wyniku do pliku
				\item Utworzenie klasy Spellcard dla Bossa
			\end{itemize}
			& \vspace{-5mm}
			\begin{itemize}
				\item Inicjalizacja urządzenia graficznego\newline Direct3D 9
				\item Trajektorie obiektów i wyliczanie przesunięcia
				\item Przekształcenia afiniczne sprajtów\newline i trajektorii
				\item Synchronizacja sprajtów z obiektami gry
				\item Wzory pocisków wrogów: kształt linii, elipsy i spirali
				\item Wczytywanie planszy gry z pliku XML
				\item Zarządzenie zasobami sprajtów
				\item Utworzenie fabryk obiektów
				\item Utworzenie parserów XML
				\item Obsługa wadliwego formatu plików\newline wejściowych
				\item Obsługa zakończenia gry po pokonaniu bossa
			\end{itemize} \\
			\hline {\large \textbf{3.~Motyka Marek}}
			&  {\large \textbf{4.~Wudecki Wojciech}} \\
			\hline 
			\vspace{-5mm}
			\begin{itemize}
				\item Utworzenie hitboxa, jego typów oraz kształtów
				\item Obsługa kolizji między hitboxami
				\item Możliwość wykrywania kolizji z elipsą
				\item Wykrywanie oraz zwiększanie otarć\newline między obiektami
				\item Utworzenie i obsługa wszystkich bonusów
				\item Realizacja zmian po zderzeniu z bonusem
				\item Przyciąganie bonusów ku graczowi
				\item Usuwanie bonusów z pamięci
				\item Implementacja DirectInput i reakcji na wciskanie klawiszy
				\item Możliwość definiowania własnych klawiszy
				\item Wyświetlanie napisów w grze (klasa Font)
			\end{itemize}
			& \vspace{-5mm}
			\begin{itemize}
				\item Narysowanie i napisanie ekranu powitalnego
				\item Utworzenie klasy nadrzędnej Playfield jako miejsca, gdzie mogą być wyświetlanie elementy gry
				\item Utworzenie wrogów, ich klasy i sprajtów
				\item Realizacja zależności pomiędzy wrogami, a ich wzorami pocisków
				\item Strzelanie pociskami przez wrogów
				\item Generowanie bonusów przez wrogów
				\item Narysowanie tła
				\item Wyświetlanie wyników w podmenu Scores
				\item Możliwość zmiany ustawień w podmenu Options
				\item Resetowanie ustawień do wartości\newline domyślnych
			\end{itemize} \\
			\hline 
		\end{tabular} 
	\end{center}
	