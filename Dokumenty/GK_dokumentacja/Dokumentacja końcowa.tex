% !TeX spellcheck = pl_PL
\documentclass[a4paper,twoside]{article}
\usepackage{polski}
\usepackage[utf8]{inputenc}
\usepackage{graphicx}
\usepackage{amsmath}

\usepackage[unicode, bookmarks=true]{hyperref} %do zakładek
\usepackage{tabto} % do tabulacji
\NumTabs{6} % globalne ustawienie wielkosci tabulacji
\usepackage{array}
\usepackage{multirow}
\usepackage{array}
\usepackage{dcolumn}
\usepackage{bigstrut}
\usepackage{color}
\usepackage[usenames,dvipsnames]{xcolor}
\usepackage{svg}
\usepackage{xfrac}
\usepackage{floatrow}
\usepackage{enumitem}
\usepackage{listings,lstautogobble}


\usepackage{multirow,tabularx}
\newcolumntype{Y}{>{\centering\arraybackslash}X}
\renewcommand{\arraystretch}{2}

% === Reset inkrementacji sekcji przy nowym parcie === %
\usepackage{titlesec}

\makeatletter
\@addtoreset{section}{part}
\makeatother
\titleformat{\part}[display]
{\normalfont\LARGE\bfseries\centering}{}{-60pt}{}

% === Dodanie krpki do sekcji
\titlelabel{\thetitle.\quad}


\setlength{\textheight}{24cm}
\setlength{\textwidth}{15.92cm}
\setlength{\footskip}{10mm}
\setlength{\oddsidemargin}{0mm}
\setlength{\evensidemargin}{0mm}
\setlength{\topmargin}{0mm}
\setlength{\headsep}{5mm}

\setlength{\textfloatsep}{10pt plus 1.0pt minus 2.0pt}


\begin{document}
	\bibliographystyle{plain}
	
	% ************************************************************
	% --- Strona tytułowa
	% ************************************************************
	\begin{titlepage}
		\begin{table}[htbp]
			\centering
			\begin{tabular}{|c|c|c|c|c|c|c|}
				\hline
				\multicolumn{7}{|c|}{\textbf{{\LARGE Grafika Komputerowa}}} \bigstrut\\[4pt]
				\hline
				Rok akademicki & Termin & Rodzaj studiów & Kierunek & Prowadzący & Grupa & Sekcja \bigstrut\\
				\hline
				\multicolumn{1}{|c|}{\multirow{2}[4]{*}{{\large 2014/2015}}} & \multicolumn{1}{c|}{{\large Piątek}} & \multicolumn{1}{c|}{\multirow{2}[4]{*}{{\large SSI}}} & \multicolumn{1}{c|}{\multirow{2}[4]{*}{{\large INF}}} & \multicolumn{1}{c|}{\multirow{2}[4]{*}{\begin{tabular}{@{}c@{}}{\large dr} \\[-9pt] {\large Ewa Lach}\end{tabular}}} & \multicolumn{1}{c|}{\multirow{2}[4]{*}{{\large GKiO3}}} & \multicolumn{1}{c|}{\multirow{2}[4]{*}{{\large 1}}} \bigstrut\\
				\cline{2-2}    \multicolumn{1}{|c|}{} & \multicolumn{1}{c|}{{\large 09:30 - 11:00}} & \multicolumn{1}{c|}{} & \multicolumn{1}{c|}{} & \multicolumn{1}{c|}{} & \multicolumn{1}{c|}{} & \multicolumn{1}{c|}{} \bigstrut\\
				\hline
			\end{tabular}%
		\end{table}%
		
		\centering
		\includegraphics[width=0.6\textwidth]{./images/logo.png}
		\\\vspace{10mm}
		\textbf{{\huge Dokumentacja końcowa}}\\\vspace{5mm}
		\textbf{{\Huge Danmaku Shooter}}
		\\
		\vfill
		\begin{flushright}
			{\Large \textbf{Skład sekcji}:}\\
			\begin{tabular}{rr}
				{\Large Buchała} & {\Large Bartłomiej}\\[-3pt]
				{\Large Forczmański} & {\Large Mateusz}\\[-3pt]
				{\Large Motyka} & {\Large Marek}\\[-3pt]
				{\Large Wudecki} & {\Large Wojciech}
			\end{tabular}
		\end{flushright}
		
	\end{titlepage}
	
	\part{\huge \textbf{Treść zadania}}
	
	\newpage
	\part{\huge \textbf{Analiza zadania}}
		\section{Sprawdzanie kolizji - okrąg czy elipsa?}
			W trakcie naszej pracy nad programem początkowo wykorzystywaliśmy okrąg do sprawdzania kolizji. Wysokość i szerokość wszystkich sprajtów była w przybliżeniu równa, więc punkty środkowe okręgu i sprajta były równe, a promieniem był krótszy z boków. Jednak po czasie pojawiły się kształty (np. bomba), dla których obsługa kolizji poprzez okrąg wyglądałaby nieatrakcyjnie. Wówczas postanowiliśmy wykorzystać elipsę. Okrąg jest jedynie specjalnym przypadkiem elipsy o równych półosiach. Gdy obiekt miał mieć hitbox w kształcie elipsy, dłuższy bok definiował długość jednej półosi, a krótszy drugiej.\\\\
			Rozwiązanie to okazało się bardzo mało efektywne - do wykrywania kolizji potrzebny były dodatkowy parametr, kąt pomiędzy obiektami. Dla jednego testu należało obliczyć ten kąt, następnie obliczyć długość promienia elipsy dla tego kąta u obu obiektów i dopiero wtedy była znane zajście kolizji. Dla hitboxów o równych osiach powodowało to mnóstwo niepotrzebnych obliczeń. Nawet pominięcie ich w przypadku równych półosi wymagało obliczania i przekazania kąta, który dla każdej elipsy był niezbędny.\\\\
			W takiej sytuacji mieliśmy dwie możliwości:
			\begin{enumerate}
				\item \textit{Zmiana elipsy wieloma okręgami} - eliminuje wykorzystanie elipsy, jednak dla każdego obiektu jest potrzebne zdefiniowanie ile okręgów potrzebuje, w którym miejscu i o jakiej wielkości, tak, by dopasować je do kształtu sprajta. Projektowanie ciała okręgów byłoby niezwykle niewygodne bez wsparcia designerskiego, którego nasza gra nie miała. 
				\item \textit{Osobne klasy dla okręgu i elipsy} - okręgu potrzebowałyby jak najmniejszej liczby danych i obliczeń, a elipsa pozostałaby obsługiwana. Wadą jest naruszenie zasad polimorfizmu - przy każdym teście okręgi nie potrzebują kąta między obiektami, więc przekazywanie ich jest zbędne i prowadzi do niepotrzebnych operacji.
			\end{enumerate}
		\section{Sprawdzanie kolizji - zastrzyk zależności}
			Rozwiązaniem problemu opisanego w powyższym rozdziale okazał się wzorzec projektowy, zastrzyk zależności (\emph{dependency injection}). Realizujemy go w ten sposób, że przy każdym teście kolizji, do jednego hitboxa wstrzykujemy drugi - pierwszy zna swoje metody działania, rzutuje drugi na odpowiedni typ i sprawdza kolizję zwracając true lub false.\\\\
			Rozwiązanie jest skuteczne, ponieważ jeżeli w teście kolizji znajduje się elipsa, to pobiera one dane jakiej jej potrzeba. Jeżeli nie, test wykonywany jest najmniejszym możliwym kosztem. Sprawdzaniem, z którym typem hitboxa mamy do czynienia zajmuje się mechanizm RTTI.
	
	\newpage
	\part{\huge \textbf{Podział pracy}}
	
	\newpage
	\part{\huge \textbf{Specyfikacja zewnętrzna}}
	
	\newpage
	\part{\huge \textbf{Przykład działania}}
	
	\newpage
	\part{\huge \textbf{Specyfikacja wewnętrzna}}
	
	\newpage
	\part{\huge \textbf{Testowanie i uruchamianie}}
	
	\newpage
	\part{\huge \textbf{Wnioski}}
	
	
	
	
	
	
	
	
	
	
	
	
	
	
	
	
	
	
	
\end{document}